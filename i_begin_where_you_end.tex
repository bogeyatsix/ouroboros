\documentclass[11pt]{letter}
\usepackage{xunicode}
\usepackage{xltxtra}
\defaultfontfeatures{Mapping=tex-text}
\usepackage{fontspec, graphicx}
\usepackage[dvipdfm, colorlinks, breaklinks, pdftitle={Stolen},pdfauthor={Loh, Adrian}]{hyperref}
\usepackage[usenames]{color}
\definecolor{Gray}{rgb}{.7,.7,.7}
\newcommand{\smallprint}[1]{\fontspec{Adobe Caslon Pro}\fontsize{10pt}{13pt}\color{Gray}\selectfont #1}
\newcommand{\ga}[1]{\fontspec[Ligatures={Common}]{Adobe Garamond Pro}\fontsize{48pt}{50pt}\selectfont #1}
%\usepackage{setspace}
%\doublespacing
\begin{document}
I begin this from where you ended, from the wishes of your 28th birthday, from
your core of absolution, from your bed of hope. I thought about the angels of my
own life –- sent to me as I stood breathless upon those shivering intersections
–- how they’ve taught me to dance to the beat of a different drum, to hear the
lost rhythms and the abandoned melodies of youth; amidst the cacophony of the
world, to listen –- for the lonesome waltz and the music from a farther room.

I guess I’ve been blessed similarly in that way. We’ve been blessed. How we have
become the accumulation of those experiences, of those recurring hopes and
dreams, the convergence of those rivers of tears. My favorite passage from \emph{The
English Patient} reads:
\begin{quotation}
	\emph{“We die, we die rich with lovers and tribes, tastes we
	have swallowed, bodies we have entered and swum up like rivers, fears we have
	hidden in, like this wretched cave. We are the real countries, not the
	boundaries drawn on maps with the names of powerful men. I know you will come
	and carry me out into the palace of winds. That's all I've wanted— to walk in
	such a place with you, with friends, on an earth without maps.”}
\end{quotation}

\emph{“Random”} you’ve always said. We are but temporary arrangements, transient and
isolated and burning in every moment with no before or after. We are love’s
echo. I always liked those snapshots at the end of Before Sunrise, when you see
the streets they walked the night before empty in the morning, it was funny that
following Sunday when I went to get my car from Sunway after you picked me up
the night before, going through all the same roads, like a \emph{déjà vu} one gets
after really long weekends. How tomorrow, the light will come to reclaim us all,
our voices and our whispers, our bodies, our stories.

A long time ago, I flirted with the idea of doing a photographic series on the
spaces that have mattered to me in my life: the swing in the garden of my old
house, that sunlit corner in my primary school where my grandma used to bring me
lunch, that pathway Anthony and I used to walk to get to Petaling Street from
school, that park bench where I made out with my first girlfriend, the departure
gates of KLIA where I said a thousand goodbyes and felt my heart sink into a
vacuum every time, those mindless wanderings down Jalan Ulu Klang at four in the
morning. I thought about a while back we were talking about the identity of
Kuala Lumpur, and how any significant artistic work must encapsulate the
mythology of the spaces of our lives. The purpose of any collective artistic
movement must serve the invention of mythology. The aggregation of its origins,
its histories, its rituals of birth, marriage and dying.

I guess that’s why I make it a point to see my friends for lunch in the city, as
I have enjoyed tremendously the last few weeks meeting you in the city, the
chance to walk those “old streets” again, to listen to the city again, to assign
significance to these spaces again, the same streets we walk now and the secret
narrative histories they contain for each of us. I feel as writers, we have to
do this if we are to ever draw stories from them. To live with a heightened
sense of things. The romantic image of strangers across a street, lovers when
you sleep. So much is left to chance. To serendipity. So much depends on it. How
often I have found more solace in the company of strangers than in the arms of
lovers. How often in the arms of a lover I have found the navel of the earth,
the return, home.

If you google the phrases \emph{“I should have”} and \emph{“I shouldn’t have”} (and for added
fun tag in the last word \emph{“kissed”}) you will see that humanity’s regrets for
those things we should have done, far outnumber and outweigh the regrets we have
for the things we did do. I fear that I will live having not lived enough. Or
love and not loved enough. I guess I stand now at the union of those fears and
realizations. I used to ponder a lot about destiny. But I’ve come to learn that
destiny takes too long. There is that creeping voice that whispers “If not now,
then when?”. I used to joke that on my gravestone, my epitaph will read: \emph{“Now
what?”}. Maybe now it should be: \emph{“Where’s the party, dood?”}.

It was the cessation of stories that irked me most. For a long time, I stopped
chasing stories, I stopped inventing memories. My heart became autistic,
dyslexic. If \emph{“the heart is an organ of fire”}, then somewhere along the way, I
let my heart atrophy. I remember one of the first times I wrote to you I wrote
that I felt like I had become a broken Richter Scale, no longer able to register
tremor and reverberation. I guess the last few months since then have been a
crucible of sorts. We used to think that the idea was to emerge unscathed with
our hearts intact. But I’m glad I’ve emerged (emerging?) scathed, bleeding with
my heart on fire.

Beckett once started a famous essay with the statement: \emph{“The danger lies in the
neatness of identifications”}. I used to chant it to myself like a mantra when I
wrote, when I thought about stories, and about people. I find it amusing how you
and I are always playing the ‘I’ and ‘E’ thing (you still owe me two more
alphabets by the way) even though on certain nights it would seem completely the
other way around. I find myself falling in love with people again. With their
stories, their truths, their deceptions, their hopes, their illusions. I’ve come
to rediscover the two great truths that have always driven my search: that
nothing is ever only one thing. And that there is nothing that does not yearn.
We are fugitives of emptiness, of absence. Our lives and our dreams are shaped
by those things we are without. I am drawn eternally to those fault-lines of the
heart, to its imperceptible shifts, its restless velocities and its involuntary
convulsions betrayed in a moment of fire and surrender, to the gravity of its
unyielding fulcrums, the autonomic faith of the spirit to find that which will
outlast Shakespeare. To emerge perhaps, one day, into the garden of that first
spring light, into that \emph{“palace of winds”}, where nothing obstructs vision except
the limits of vision itself.

I thank God --- and truly I do --- for convergences, for chance, for the roll of
the dice and the luck of the draw, for the angels I have found in the
wilderness, and those thrust upon me in the eye of storms. I end here where you
ended, upon your hopes that when you check out, that you would have given more
than you have taken. And I write this now to thank you, to let you know (because
sometimes we have to know) that you have given, and been all this to me. The
listening presence in my silence, the harbors of my forgotten restlessness, the
secret sharer. I thank you for having helped me co-author the prologue of these
quiet revelations of the last few months. For truly, they would not have been
possible without you (or at least, it would have taken me much longer to arrive
at them). And finally, for taking me along into that frenetic random world of
yours at a time when I needed most to be reminded what was missing from mine. I
would have kicked myself silly had I been told then that I’d be writing such a
thing as this one day (to a girl who went to GIS no less!). OK, you can get off
that pedestal now. Turn off the lights, children. Our play is played out.
Tomorrow we’ll turn back into pumpkins and wonder about the visions we have yet
to dream and the stories we have yet to tell.
\end{document}